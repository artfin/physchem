\documentclass[14pt]{article}

\usepackage{amsmath}
\usepackage{mhchem}

\usepackage{fullpage} % Package to use full page
\usepackage{parskip} % Package to tweak paragraph skipping
\usepackage{tikz} % Package for drawing
\usepackage{amsmath}
\usepackage{hyperref}
\usepackage{gensymb} % for degree symbol

\usepackage[T3]{fontenc}
\usepackage[utf8]{inputenc}
\usepackage[russian]{babel}

\usepackage{float}
\usepackage{chemfig}

% defining newcolumntype
\usepackage{array}
\newcolumntype{P}[1]{>{\centering\arraybackslash}p{#1}}

\title{Вращательная спектроскопия. \\ Домашнее задание.}
\author{Финенко Артем}
\date{2016/28/10}

\begin{document}

\maketitle

\section*{Задача 7.1.}

\begin{gather}
\ce{{}^6 LiF} \notag \\ 
E_r^{(1)} = B \ J (J+1) \bigg{|}_{J=1} = 2B \notag \\
E_r^{(2)} = B \ J (J+1) \bigg{|}_{J=0} = 0 \notag \\
\Delta E_r = 2B \quad \implies \quad B = \frac{\Delta E_r}{2} = 1.497 \ cm^{-1} \notag \\ 
l = \sqrt{\frac{h}{8 \pi^2 c \mu B}} = 1.57 \text{\AA} \notag
\end{gather}

\section*{Задача 7.2.}
Предполагаем, что длины связи в $\ce{{}^6 LiF}$ и $\ce{{}^7LiF}$ совпадают.
\begin{gather}
B_1 = \frac{h}{8 \pi^2 c \mu_1 l^2} \quad \implies \quad B_2 = \frac{h}{8 \pi^2 c \mu_2 l^2} = B_1 \frac{\mu_1}{\mu_2} \notag \\
B_1 = 89740.46 \ MHz \quad \implies \quad B_2 = 79997.21 \ MHz \notag
\end{gather}

\section*{Задача 7.5.}
Изначально поместим начало отсчета в атом $\ce{H}$ молекулы цианоацетилена и определим положение центра масс.
\begin{gather}
x_{COM} = \frac{\sum_i m_i x_i}{\sum_i m_i} = 3.00 \text{\AA} \notag
\end{gather}

В системе отсчета, связанной с центром масс, найдем главные компоненты тензора инерции.
\begin{gather}
I_{aa} = \sum_i m_i (y_i^2 + z_i^2) = 0 \notag \\
I_{bb} = \sum_i m_i (x_i^2 + z_i^2) = 1.8597 \cdot 10^{-45} \ kg \cdot m^2 \notag \\
I_{cc} = \sum_i m_i (x_i^2 + y_i^2) = I_{bb} \notag \\
I_{bb} = I_{cc} > I_{aa} = 0 \notag
\end{gather}

Найдем вращательную постоянную $B$.
\begin{gather}
B = \frac{h}{8 pi^2 c I} = 0.1505 cm^{-1} = 4.513 \ GHz \notag
\end{gather}

\section*{Задача 7.6.}
\renewcommand{\arraystretch}{1.6}
\begin{table}[H]
\begin{center}
\begin{tabular}{|P{1.5cm}|P{2cm}|P{2cm}|P{2cm}|P{2cm}|}
\hline
& $I_{aa}, kg \cdot m^2$ & $I_{bb}, kg \cdot m^2$ & $I_{cc}, kg \cdot m^2$ & $B, GHz$  \\ \hline
$\ce{CO2}$ & 0 & $7.65 \cdot 10^{-46}$ & $7.65 \cdot 10^{-46}$ & 10.967 \\ \hline
$\ce{OCS}$ & 0 & $1.459 \cdot 10^{-45}$ & $1.459 \cdot 10^{-45}$ & 5.749 \\ \hline
$\ce{O{}^{13}CS}$ & 0 & $1.464 \cdot 10^{-45}$ & $1.464 \cdot 10^{-45}$ & 5.731 \\ \hline
$\ce{CS2}$ & 0 & $2.71 \cdot 10^{-46}$ & $2.71 \cdot 10^{-46}$ & 3.085 \\ \hline
\end{tabular} 
\end{center}
\end{table}    

Замена $\ce{{}^12C}$ на $\ce{{}^13 C}$ приводит к изменению вращательной постоянной только в случае $\ce{OCS}$. 

\section*{Задача 7.8.}
\begin{center}
\ce{{}^{14} N {}^{16} O_2 {}^{35} Cl} \\ 
[\baselineskip]
\chemfig{(O-[1]N(-[7]O)(-[2]Cl))}
\end{center}

Определяем положение центра масс и относительно него вычисляем координаты атомов. Затем определяем главные компоненты тензора инерции и вращательные постоянные.

\begin{table}[H]
\begin{center}
\begin{tabular}{|P{1.5cm}|P{3cm}|P{3cm}|P{3cm}|P{1.5cm}|P{1.5cm}|P{1.5cm}|}
\hline
 & $I_{aa}, kg \cdot m^2$ & $I_{bb}, kg \cdot m^2$ & $I_{cc}, kg \cdot m^2$ & A, GHz & B, GHz & C, GHz \\ \hline
 \ce{NO2Cl} & $1.614 \cdot 10^{-45}$ & $6.336426 \cdot 10^{-46}$ & $2.248106 \cdot 10^{-45}$ & 5.198 & 13.244 & 3.733 \\ \hline
\end{tabular}
\end{center}
\end{table} 


\end{document}
